% Home Network Diagram
% Author: Michael Bugert
% License: CC BY 4.0
\documentclass{standalone}

\usepackage[utf8]{inputenc}
\usepackage[sfdefault]{noto}
\usepackage[T1]{fontenc}
\usepackage{enumitem}
\usepackage[fixed]{fontawesome5}
\usepackage{etoolbox}
\usepackage{tcolorbox}
\usepackage{tikz}
\usepackage[hidelinks]{hyperref}

\tcbuselibrary{skins,raster}
\usetikzlibrary{positioning,matrix,fit,decorations.pathreplacing}

\graphicspath{{img/}}

% colors for subnetworks
\definecolor{subnet-none}{HTML}{000000}
\definecolor{subnet-home}{HTML}{E57400}
\definecolor{subnet-guest}{HTML}{14C300}
\definecolor{subnet-vpn}{HTML}{AD0022}
\definecolor{docker-bright}{HTML}{0A97C4}
\definecolor{docker-dark}{HTML}{088AB4}

\tcbset{
	skin=enhanced,
	boxrule=0.1em,
	left=0.25em,
	right=0.25em,
	% box with hostname and device specs
	host description box/.style={
		width=19em,
		sidebyside,
		righthand ratio=0.4,
		fonttitle=\bfseries\large,
		fontupper=\small,
		fontlower=\small
	},
	% tcbraster for network interfaces
	interfaces raster/.style={
		raster columns=1,
		raster width=6.75em,
		% fix raster height to accomodate for fontawesome icon height differences
		raster height=4.25em,
		raster row skip=0.4em,
		raster every box/.style={
			fontupper=\ttfamily,
			left=0em,
			right=0em,
			top=.15em,
			bottom=.15em,
			valign=center
		}
	},
	% tcbraster for host services
	services raster/.style={
		raster columns=1,
		boxrule=0.1em,
		raster width=14em,
		raster row skip=0.4em,
		raster every box/.style={
			left=0em,
			right=0em,
			top=.15em,
			bottom=.15em,
			fontupper=\small,
			boxrule=0.1em
		}
	},
	% box around host services:
	% Never drawn, only used as a container. Width is set to the same width as "services". If width is not specified, the box would extend much further towards the right because the default width is \linewidth (uncomment 'blankest' to see it).
	services box/.style={
		blankest,
		width=14em
	},
	% box around dockerized host services
	dockerized box/.style={
		%title={Dockerized\hfill\includegraphics[height=12pt]{services/docker}},
		title={Dockerized},
		fonttitle=\bfseries,
		colframe=docker-dark,
		colback=docker-bright,
		left=0.25em,
		right=0.25em
	},
	% box around devices with less details
	devices box/.style={
		width=20em,
		fonttitle=\bfseries\large,
		fontupper=\small,
		attach boxed title to top left,
		boxed title style={frame hidden, opacityback=0.0, left=0em},
		colback=white,
		coltitle=black!75!white
	},
	% tcbraster for devices with less details
	devices raster/.style={
		raster columns=1,
		boxrule=0.1em,
		raster width=20em,
		raster row skip=0.4em,
		raster every box/.style={
			left=0em,
			right=0em,
			top=.15em,
			bottom=.15em,
			fontupper=\small,
			boxrule=0.1em,
			height=3em,
			valign=center
		}
	}
}

\tikzset{
	single icon/.style={
		font={\fontsize{34}{38}\selectfont}
	},
	host description/.style={
	},
	% box drawn around all bits defining a host using tikz fit
	host box/.style={
		inner sep=0.5em,
		draw=black!75!white,
		fill=white,
		rounded corners,
		line width=0.1em
	},
	% box drawn around all bits defining a region using tikz fit
	region box/.style={
		inner sep=2em,
		rounded corners=2em,
		line width=0.1em,
		fill=#1!5!white
	},
	region descriptor/.style={
		outer sep=0,
		execute at begin node={\fontsize{7em}{7em}\selectfont\scshape},
		color=#1!15!white
	},
	% tag for marking devices or services as DNS, DHCP, etc. with a customized color
	tag/.style={
		draw=none,
		fill=#1,
		rectangle,
		rounded corners=2pt,
		outer sep=3pt,
		font=\bfseries\scriptsize,
		text=white
	},
	% connection types
	generic/.style={
		solid,
		line width=0.2em,
		rounded corners
	},
	ethernet/.style={
		generic
	},
	wifi/.style={
		generic,
		dashed,
	},
	vpn/.style={
		generic,
		decoration={
			waves,
			radius=0.8em,
			segment length=.55em,
			angle=30
		}
	}
}

% commands to show device and service icons at fixed size, etc.
\newcommand{\interfaceImage}[1]{%
	\raisebox{-.1\height}{\includegraphics[height=12pt]{#1}}%
}
\newcommand{\deviceImage}[1]{%
	\includegraphics[width=5em, height=3em, keepaspectratio]{devices/#1}%
}
\newcommand{\deviceImageInline}[1]{%
	\raisebox{-.25\height}{%
		\parbox[t]{2em}{%
			\centering%
			\includegraphics[width=2em, height=2em, keepaspectratio]{devices/#1}%
		}%
	}%
}
\newcommand{\ip}[2]{%
	\textbf{\textcolor{#2}{#1}}
}

% clickable URLs shown without protocol
\newcommand{\https}[1]{%
	\href{https://#1}{\nolinkurl{#1}}%
}
\newcommand{\http}[1]{%
	\href{http://#1}{\nolinkurl{#1}}%
}

% usage: \service{image}{textual description}{url}
\newcommand{\service}[3]{%
	\ifblank{#1}{%
		\hspace{2em}%
	}{%
		\raisebox{-.3\height}{%
			\includegraphics[height=2em]{services/#1}%
		}\hspace{.5em}%
	}%
	\parbox[m]{10em}{%
		#2%
		\ifblank{#3}{}{%
			\newline%
			\tiny\textcolor{gray}{#3}%
		}%
	}
}

\newcommand{\planned}{\hfill\scriptsize\textcolor{black!60}{\textit{(planned)}}}
\newcommand{\temporary}{\hfill\scriptsize\textcolor{black!60}{\textit{(temporary)}}}

\begin{document}
	
% remember picture is needed to retain positions of tcolorboxes (otherwise all edges will be misplaced)
\begin{tikzpicture}[remember picture]
	\pgfdeclarelayer{regions}
	\pgfdeclarelayer{hostboxes}
	\pgfsetlayers{regions,hostboxes,main}
	

	% --------------- start PIO ---------------
	% host description
	\node[host description] (router-description) {%
		\begin{tcolorbox}[
			adjusted title=PIO,
			host description box
		]			
			\ip{10.1.1.31}{subnet-home}
		
			\tcblower
			
			GN-9373
		\end{tcolorbox}
	};
	\node[
		below=0em of router-description.south west,
		anchor=north west
	] (laptop-services) {
		\begin{tcboxedraster}[services raster]{services box}
			\tcbox{\service{}{bilgecontrol}{}}
		\end{tcboxedraster}
	};
	
	% services
	\node[
		below=0em of router-description.south west,
		anchor=north west
	] (router-services) {
	};
	
	% network interfaces
	\node[anchor=north] at (router-services.north east -| router-description.east) (router-interfaces) {
		\begin{tcbitemize}[interfaces raster]
			\tcbitem[remember as=router-eth0] eth0..1\hfill\faIcon{ethernet}
		\end{tcbitemize}
		
	};
	
	% draw box around all parts defining a host
	\begin{pgfonlayer}{hostboxes}
		\node[
			host box,
			fit=(router-description.center) (router-services.south) (router-services.west) (router-interfaces.south)
		] (router-host-box) {};
	\end{pgfonlayer}
	% --------------- end router ---------------
	


	

	% --------------- start raspberrypi ---------------
	% host description
	\node[host description, right=8cm of router-host-box] (raspberrypi-description) {%
		\begin{tcolorbox}[
			adjusted title=Bargecontrol network,
			host description box
		]
		IP: \hspace*{\fill} \ip{10.1.1.X}{subnet-home}\\
		 SN: \hspace*{\fill} \ip{255.255.255.0}{subnet-home} \\
		Gateway: \hspace*{\fill} \ip{10.1.1.1}{subnet-home}\\
		VLAN: \hspace*{\fill} \ip{801}{subnet-home}
		\vspace{\baselineskip}
		
		\begin{tabular}{@{}ll@{}}
			\faIcon{compact-disc} & Ubuntu 18.x.x\\
		\end{tabular}
		
		\tcblower
		
		
		Industrial PC
		\end{tcolorbox}
	};
	
	% services
	\node[
		below=0em of raspberrypi-description.south east,
		anchor=north east
	] (raspberrypi-services) {
		\begin{tcboxedraster}[services raster]{services box}
			\tcbox[enhanced, remember as=raspberrypi-pihole]{\service{}{PLCHandler}}
		\end{tcboxedraster}
	};

	
	% network interfaces
	\node[anchor=north] at (raspberrypi-services.north west -| raspberrypi-description.west) (raspberrypi-interfaces) {
		\begin{tcbitemize}[interfaces raster]
			\tcbitem[remember as=raspberrypi-eth0] \faIcon{ethernet}\hfill eth0
		\end{tcbitemize}
	};
	
	% draw box around all parts defining a host
	\begin{pgfonlayer}{hostboxes}
	\node[
		host box,
		fit=(raspberrypi-description.center) (raspberrypi-services.south) (raspberrypi-services.east) (raspberrypi-interfaces.south)
	] (raspberrypi-host-box) {};
	\end{pgfonlayer}
	% --------------- end raspberrypi ---------------
	
	% --------------- start guest devices ---------------
	\node[host description, below=4em of router-host-box.south, anchor=north, inner sep=0em] (guest-devices) {
		\begin{tcboxedraster}[devices raster]{%
			devices box,
			title={Guest Devices\hspace{1em}\ip{10.1.1.2/18}{subnet-guest}},
			remember as=guest-devices-raster}
			\tcbox{Service laptop}
		\end{tcboxedraster}
	};
	% --------------- end guest devices ---------------


		

	\begin{scope}[ethernet]
		\draw[subnet-home] (raspberrypi-eth0.west) -- ++(-1,0) |- (router-eth0.east);
		\draw[subnet-home] (guest-devices.east) -- ++(2,0) |- (router-eth0.east);
	\end{scope}

	% --------------- end network connections ---------------
\end{tikzpicture}
	
\end{document}